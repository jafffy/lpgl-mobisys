\section{Discussions and Future Work}

%Based on our experience, we outline several interesting points for 
%further discussion and needing future investigation.

%In this work, we defined scene dynamics metric as the maximum speed of visible objects, which is faster then image-based method(e.g. PSNR) and granularity-free.
%To realize geometry-based method, the underlying system should be able to access transformation information of objects. {\myit} can capture transformation by Front-end API without any additional overhead.
%However geometry-based dynamics scoring cannot work for image based application like 360$\circ$ video streaming application.

\spar{Applicability to other mobile AR headsets:}
%
{\myit} is designed to be a device and platform-independent solution for reducing power usage on mobile AR headsets. Given that its design is centered around the standard graphics pipeline and native graphics library APIs (e.g., OpenGL), the implementation is easy to port for different platforms. In this work, while we mostly focus on the performance statistics of the {\mlo}, as the results in Section~\ref{sec:heat} shows, {\myit} is already implemented for use on the Microsoft Hololens as well. We note that we observed similar power usage reduction patterns on the Hololens by saving $\sim25\%$ of power for static scenes and $\sim12\%$ for dynamic scene applications. As part of our future work, we plan to expand {\myit} support for different mobile AR platforms as well.

%Furthermore, since {\myit} focuses on object characteristics, while designed for AR apps, we see potential in applying similar techniques to VR headsets as well.

% this is where either we have already shown it with Hololens results or we give some Hololens results. -- emphasize that this scheme is generally applicable...

%{\myit} is applicable to other another AR headsets like Microsoft HoloLens. We implemented {\myit} on HoloLens and conducted energy consumption measurement on static scene scenario aforementioned. As similar with Magic Leap One, {\myit} could save x\% of power consumption of HoloLens as well.
%
%{\myit} is portable to graphics-based platform by reason of its minimal assumption. {\myit} requires only standard graphics pipeline and the most of modern mobile platform which have display and GPU provides graphics API like OpenGL or DirectX. By this characteristic {\myit} might be portable to general mobile phone or VR devices, but there should be preceded investigation on energy consuming characteristic of each devices.

\spar{Power management of headset and other components:}
%
While our preliminary studies show that the headset itself is also a major consumer of power, this work focuses on the graphics pipeline-related aspects of mobile untethered AR platforms. Techniques such as those that duty-cycle the headset sensors and adaptively control the brightness/resolution of objects, combined with our efforts in optimizing the graphics pipeline, can have significant impact in increasing the lifetime of mobile AR headsets. However, the fact that the {\mlo}, along with many other commercially available mobile AR headsets, expose only limited access to the headset configurations limits such research at this point. 

In addition, the power usage of components such as WiFi need to be collectively managed. As mentioned in Section~\ref{sec:preliminary}, findings from many previous work~\cite{Qian2018, Qian2018TVP, Abari2017} can be applied to reduce WiFi power usage in mobile AR applications. Overall, we see this as an interesting direction of research towards designing a comprehensive framework for mobile AR headset power management. 


\spar{User study limitations: }
%
Given that mobile AR headset development is still in its early stages, real-world applications for exploiting these platforms' full capabilities are still premature. At the same time, considering the impact that these new wearable computing devices can offer, the domains in which they can potentially contribute to can become very diverse. For this reason, while we wanted to test {\myit} on a real-world, widely used application, it was difficult to define a single set of applications that were representative for mobile AR headsets. For this reason, our user studies focus on three types of applications, each with different application and scene characteristics. We do so to make sure that our evaluations cover diverse cases in how mobile AR headsets can contribute in novel applications. Nevertheless, the performance of {\myit} and untethered mobile AR headsets in general will heavily depend on application complexity and its requirements, and we hope to apply {\myit} on real-world applications as part of our future work.

%\spar{Additional Improvements:}
%
%A core feature of {\myit}, frame rate control, is activated using the scene dynamics score computed based on the maximum speed of objects moving on the scene. While effective, there are limitations. Given our scheme focuses on the ``movements'' computed using the bounding-box of the object (to minimize computational complexity), if for example a sphere-shaped globe was to rotate 360$^\circ$ without coordinate changes, {\myit} would not be able to capture such dynamics. This is a fundamental limitation of geometry-based schemes that compute the scene dynamics and as part of future work we plan to explore additional scene dynamics computing methods. 

%On the system's perspective, along with the technologies presented in this work, other power consuming components, such as the display and WiFi. As briefly mentioned in Section~\ref{sec:prelim}, the omit the control of these components due to either the limitations in hardware configurations or 






%We perform our user studies with three different types of applications to show the effectiveness of {\myit} in diverse scenarios. However, g

%In designing user studies, it was hard that defining ``realistic'' and ``representative'' application of AR headset. 1) \textit{pre-maturity of ecosystem} Since AR headsets and their applications are not yet adopted to common users,  2) \textit{Generality}

%Not yet AR headset application ecosystem is not mature

%AR application boundary is too broad.

%\jw{Should include consideration for people who have bad sight.}

% \spar{Power Management Impact on Heat and Comfort}

% The most of participants complained that the heat of headset and it is potentially able to cause user experience quality degradation.
% Unfortunately, the Power and Thermal Analyzer, which is provided by Maigc Leap~\cite{}, does give the temperatures of each components in the board (e.g. CPU, GPU, SoC and so on), but not offer the heat information of the headset.

\begin{comment}
%
{\myit} uses quad-tree-based dynamics scoring method. 
%
Evaluation results suggest that this scheme is energy efficient for screen
contents with a limited number of 3D objects to display.
%
Compared to traditionally widely-used methods such as PSNR and SSIM which
require the entire frame buffer content, our scheme which uses only the 
geometric information is appealing.
%
However, %our results also suggest that a complex display content can result in
%high computational \emph{latency} as well.
%
while the PSNR and SSIM methods can have constant overhead, our scheme's 
\emph{latency} will scale with the number of objects on the scene.
This suggests the need for identifying the limit up to which our scheme can 
benefit over PSNR or SSIM.
\end{comment}
% \jk{limitation of geometry-based schemes is that we cannot capture changes in color/texture for non-moving objects}
%
%Furthermore, because our scheme uses the difference in bounding box coverage
%rather than the actual content when determining the changes in 
%consecutive frames, if in the unlikely case a new object is plotted in the 
%\textit{same} bounding box, the quad-tree-based scheme will not be able to 
%catch this information. We find this case very unlikely and small changes in
%the bounding box can be captured using a deep enough tree. Nevertheless,
%these are remaining issues that we see as important. 



\begin{comment}
\spar{Parameter Tuning}:
%
%Our current version of {\myit} implementation does not provide support for 
Our current version of {\myit} does not provide support for 
optimizing the parameters for a given objective.
%
Hence, we leave the ``freedom'' of configuring the parameters to the
application at the cost of demanding some level of developer involvement.
%
However, to fully achieve transparency with the application, there is a need
for {\myit} to independently understand the requirements of the application 
and also (potentially) its users.
%
On the application perspective, we are performing additional research on 
the requirement set and in the process of identifying a small number of 
preset values that satisfies such requirements.
%
On a per-user adaptation perspective, there is a need to expand user studies
to identify the diverse (minimum) quality requirements of different users.
%Through this process, there is a need to understand how {\myit} can easily
%and quickly adapt to user requirements with minimal training.
%
Indeed, the user and application requirements for parameter settings will be
a combination factor and further studies are needed to achieve the goal of
adaptive parameter auto-adjustment or simplification.



\spar{Closed System}:
%

\raj{this might be unnecessary}

Finally, we had a practical challenge of Microsoft HoloLens being closed. 
This tightly limits the services that third-party developers can provide.
%
First, HoloLens applications are sandbox-based; thus, cannot access the 
system-related information freely. In addition, the lack of internal 
inter-process communication mechanism in UWP limits the development freedom 
and the scope of services that can be designed.
%
Second, all third-party programs are considered as ``applications".
As a result, services such as {\myit} cannot run as background daemons. % or services.
Instead, we had to design {\myit} %so that the implementation stays 
close to the application layer. 
If potentially there are many applications 
interconnected with {\myit}, executing as a daemon service can be beneficial. 
%For example, LiKamWa et al.~\cite{LiKamWa:2015:SEC:2742647.2742663} also proposed a
%dedicated process for vision processing to reduce Google Glass' computation
%load. However, in closed systems, such scheme cannot be directly applied.
%
Finally, head orientation and the view matrix is (currently) the only 
type of fine-grained and real-time sensory information that we can utilize 
from the HoloLens. Despite having depth and temperature sensors on-board,
UWP does not allow applications to access their data. With more sensory data, 
there are potentials to monitor and manage the energy usage in different 
dimensions.

\end{comment}


%%%%%%%%%%%%%%%%%%%%%%%%%%%%%%%%% 80 CHAR %%%%%%%%%%%%%%%%%%%%%%%%%%%%%%%%%%%%%%



%%%%%%%%%%%%%%%%%%%%%%%%%%%!!DO NOT USE!!!!%%%%%%%%%%%%%%%%%%%%%%%%%%%%%%%%%%%%%
%%%%%%%%%%%%%%%%%%%%%%%%%%%!!DO NOT USE!!!!%%%%%%%%%%%%%%%%%%%%%%%%%%%%%%%%%%%%%

\begin{comment}

Statistics on the head direction data is as follows. The average task completion time was X $\pm$ y sec, the average change in head direction was X $\pm$ y \jk{$^\circ$/sec}, and \jk{mW and MW????}. \jw{It was just variable name. I changed it to a and b}
%Among these samples, we select to most active 25\% and the least active 25\% samples for our evaluations. \jk{Why these samples? Average is less meaningful? Extreme is better?}
% 먼저 우리는 head direction data들을 통계 내보았다. 평균적인 작업의 완료 시간은 x +- e초 걸렸고, 평균적인 head direction 변화 속도는 x+-e, 최대 폭과 최소 폭은 각각 a, b였다. 우리는 이 중에서 상위 25% 넓게 움직인 샘플, 하위 25% 조금 움직인 샘플을 골라서 에너지 소모를 측정했다. %{jw: 재생 속도를 조정하는 실험은 시간이 허락하면 진행해야 할 것 같습니다.} 재생 속도를 조정해 각각의 속도를 빠르게 /  느리게 조정하여 에너지 소모를 측정했다.

\begin{figure}[t]
    \centering
    \subfigure[Changes in frame rate when varying the frame dynamics threshold. \jk{Is this a trace??}]{
        \includegraphics[width=0.44\linewidth]{empty}
        \label{fig:framedynamics-trace}
    }
    \subfigure[Energy savings for varying dynamics thresholds.]{
        \includegraphics[width=0.44\linewidth]{empty}
        \label{fig:framedynamics-energy}
    }
    \caption{Impact of the frame dynamics threshold on the frame rate and energy usage of the HoloLens.}
    \label{fig:framedyanmics}
\end{figure}


First, we present in Figure~\ref{fig:framedynamics-trace} on how the frame rate adaptively changes with respect to the changes in the frame dynamics threshold. In Section~\ref{} we mentioned that {\myit} measures the frame dynamics score and the RDCC engine makes decisions on the frame rate of the target object. The threshold ... \jk{Did we ever say what the ``dynamics threshold" is? VERY CONFUSING...} As \fig\ref{fig:framedynamics-trace} shows, for sensitive thresholds, RDCC maintains a high frame rate of 60 fps for ... \jk{WTF.... what does a ``sensitive" threshold mean??? Where do we define this?!?!?}

% 우리는 해당 energy analysis 를 우리의 방법 세가지 각각의 방법에 대한 파라미터를 바꾸어서 에너지 분석을 시도했다.
% 첫번째 분석은 frame dynamics의 threshold를 예민하게 설정한 것과 둔감하게 설정한 두 케이스를 놓고 위의 instance들에 대해서 실험하여 이러한 결과를 얻었다. 첫번째로 framerate dyanmics를 확인해봤다 (Figure~\ref{fig:framedyanmics}). 결과는 민감한 threshold를 가진 실험은 크게 움직이는 샘플에서는 60FPS 가까이를 유지하는 것을 볼 수 있었다. 반면 조금 움직이는 케이스와 낮은 감도의 instance는 크게 움직일 때는 60FPS로 framerate이 올라갔지만 평소에는 framerate이 낮았다. 이렇게 아낄 수 있는 framerate들은 에너지를 아낄 수 있는 potential로 볼 수 있다.

Next, we take this result and confirm that this adaptive control in frame rate can help improve the system's energy efficiency. As \fig\ref{fig:frameDynamics-energy} shows, for varying sensitivities (e.g., threshold values), the overall energy efficiency improves due to the reduced frame rate for less-dynamic situations. Specifically, ... \jk{detailed numbers.}

% 다음으로 우리는 실제로 에너지 소모가 아껴지는지를 확인했다. 먼저 위에서 정의한 것과 같이 framerate scaling threshold를 정의하고 중간 수준의 움직임을 반복했을 때 에너지 소모가 얼마나 아껴지는지 확인했다. 그 결과 확실히 높은 threshold일 때는 x mW/h의 전력을 소모한 것에 반해 낮은 threshold일 경우 y mW/h 정도 소모함으로써 에너지 소모를 크게 아낀 것을 알 수 있었다.

% 다음은 mesh simplification의 resolution에 따른 에너지 소모를 확인해 보았다. 각각 octree의 depth를 2, 3, 4 늘려가면서 중간 수준의 움직임에서 얼마나 에너지 소모를 아낄 수 있는지 확인했다. 이번 결과에서는 object 수가 적기 때문에 2와 3의 경우에는 에너지 소모가 크게 다르지 않았지만 4의 경우 2와 3과는 크게 달라진 것을 확인할 수 있었다. 이것은 preliminary study에서 # of triangles가 에너지 소모에 끼치는 영향과 같은 경향성을 보였다. 

% 다음으로 더 넓은 범위로 머리를 움직인 녹화를 빠르게 재생 / 느리게 재생하여 비교하였다. 그 결과 빠르게 움직인 케이스는 x mW/h 의 전력을 소모하고 느리게 움직인 케이스는 y mW/h의 전력을 소모했다. 넓은 범위로 머리를 움직인 경우 멀리 놓은 물체들은 오랫동안 렌더링 하지 않기 때문에 에너지 소모를 줄일 수 있었다. 그에 반해 빠르게 움직인 경우 dynamics 가 증가하여 framerate이 올라가고 물체들이 cull되는 시간이 적기 때문에 에너지 소모를 줄이는데 크게 도움이 되지 않았다는 것을 알 수 있다. 

\jk{WHAT ARE THE THREE PARAMETERS? DYNAMICS THRESHOLD, REMESH COMPLEXITY and what???????}
\jw{The three parameters are dynamic threshold, remesh complexity and dynamics granularity}

\end{comment}
%%%%%%%%%%%%%%%%%%%%%%%%%%%%%%%%%%%%%%%%%%%%%%%%%%%%%%%%%%%%%%%%%%%%%%%%%%%%%%%%
%%%%%%%%%%%%%%%%%%%%%%%%%%%%%%%%%%%%%%%%%%%%%%%%%%%%%%%%%%%%%%%%%%%%%%%%%%%%%%%%

