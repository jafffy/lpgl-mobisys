\section{Related Work}
\label{sec:related}

%%%%%%%%%%%%%%%%%%%%%%%%%%%%%%%%% 80 CHAR %%%%%%%%%%%%%%%%%%%%%%%%%%%%%%%%%%%%%%

%Recent studies proposed techniques that exploit human perception information  
%(e.g., gaze/head orientation or scene dynamics) to reduce computation or energy usage
%for mobile platforms while minimizing the sacrifice in user experience~\cite{Guenter:2012:FG,Hwang:2017:RPO,He:2015:OSP}. 
%We leverage similar techniques in this work.

%%%%%%%%%%%%%%%%%%%%%%%%%%%%%%%%% 80 CHAR %%%%%%%%%%%%%%%%%%%%%%%%%%%%%%%%%%%%%%

A number of prior work %among which we can list only a subset here,
exploit human perception information to reduce computation or 
energy consumption while minimizing sacrifice in user experience on mobile headsets.
%
Guenter et al.~\cite{Guenter:2012:FG} presented a way to 
improve performance by using eye tracking information to render backgrounds 
with low resolution and focused-content with high resolution.
%
Patney et al. at NVIDIA has also announced an improved foveated rendering 
technique for VR displays which optimizes performance by reducing image 
quality in the viewer's peripheral vision based on gaze tracking~\cite{foveated_rendering}.
%
Hwang et al. proposed a PSNR-based method to reduce energy consumption using
frame rate scaling by comparing the contents of two consecutive frames in
mobile games~\cite{Hwang:2017:RPO}.
%
Tan et al. proposed FocusVR~\cite{focusVR}, which performs screen dimming and vignetting to reduce VR headset power consumption.
%
While sharing similar goals with {\myit}, these works focus on the VR environment, where background rendering plays a significant role in power usage reduction. Our work is tailored towards mobile AR headsets; thus, addresses different challenges such as those discussed in Section~\ref{sec:background}.

%\todo{Our work, however, uses a gyroscope sensor instead of eye tracker 
%(which many not be available on most commercial AR HMDs) and use geometry based 
%dynamics measurements instead of image-based comparisons to reduce the 
%computation latency.}


{\myit} resides between the application and the graphics library, providing a
transparent compatible layer that can intercept OpenGL commands.
This is similar to the approach taken by Miao et al. 
in~\cite{Miao:2016:TYG:2873587.2873603}.
However, their focus was to adapt the graphics stack to the circular 
displays of wearable smart watches and reduce the memory and display interface
traffic wasted due to non-rectangular displays, while our approach focuses on
energy efficiency.

In an attempt to reduce energy usage of the display on a mobile device without
deteriorating user perception,
%
LPD~\cite{191579} reduces the memory and display interface traffic by 
utilizing the display update information to 
% eliminate expensive memory copies of unchanged parts.
suppressing the update of unchanged parts.
% However, this approach is not applicable to a general graphics pipeline 
% because the GPU architecture requires all scene to be redrawn for each frame.
%
He et al. \cite{He:2015:OSP} makes an observation that 
% human visually-perceivable ability highly depends on the user-screen distance, and 
a reduced display resolution may still achieve the same user experience
when the user-screen distance is large. Based on this idea, 
% authors present a flexible dynamic resolution scaling system for smartphones. 
authors adopt an ultrasonic-based approach to accurately detect the user-screen
distance and make dynamic scaling decisions for maximum user experience and 
power saving.
%
Anand et al.~\cite{Anand:2011} utilize the pixel brightness to 
save significant amounts of power while preserving image qualities, and
%
FingerShadow~\cite{187123} performs local dimming for the screen areas covered
by users' fingers to save power without compromising their visual experiences.
%
% and Ma et al. \cite{6200246} characterized the power consumption of mobile games.
%
However, these work are in the context of mobile devices and do not take into
account the characteristics of untethered AR headsets.
%
% As discussed in Section~\ref{sec:preliminary}, the screen resolution may not be 
% adjustable, and the brightness and object coverage may not be the main consumers
% of energy on HMDs such as the Microsoft HoloLens.

%%%%%%%%%%%%%%%%%%%%%%%%%%%%%%%%% 80 CHAR %%%%%%%%%%%%%%%%%%%%%%%%%%%%%%%%%%%%%%