\section{Conclusion}% and Future Directions}
\label{sec:conclusion}

%%%%%%%%%%%%%%%%%%%%%%%%%%%%%%%%% 80 CHAR %%%%%%%%%%%%%%%%%%%%%%%%%%%%%%%%%%%%%%


We presented {\myit}, an OpenGL API compatible Low-power Graphics Library
for energy efficient untethered mobile AR headset application development. We first presented extensive measurements detailing the power consumption characteristics of the {\mlo}. From there, we built {\myit} to use gaze/head orientation and geometry data to obtain user perception information, and exploit this to adaptively apply frame rate scaling,  mesh simplification, and culling techniques to enhance the battery lifetime of untethered AR headsets, while minimizing losses in user  perceived scene quality. Moreover, these features are fully application-layer  transparent. Through comprehensive controlled in-lab experiments and an IRB-approved 25 participant user study, we showed that {\myit} can reduce power consumption by up to $\sim$22\%, with minimal latency and user experience impact. 

%
%{\myit} requires minimal modification to the application by hiding the 
%energy efficiency consideration from the user application development.
%%
%
%We implemented a prototype of {\myit} on the Microsoft HoloLens.
%
%

%Another major contribution of this work, and the basis of {\myit}'s design, . Our findings suggest that there is much more to be done. Tackling the graphics pipeline is just a first step in lengthening the lifetime of untethered mobile AR headsets. With more freedom to adjust specific parameters of the headset and its internal sensing units, we believe that a more rich set of applications can be enabled through mobile AR headsets.

%
%With more freedom to the pink panther and AR bunnies throughout time and space.

%Our work also includes an in-depth analysis on the energy consumption patterns
%of AR HMDs, and identified major power consuming factors that are
%software-controllable.
%
%We believe that {\myit} can contribute to expand the scope of, and open up 
%opportunities for, new and exciting AR applications.


%As part of future work, we plan to expand {\myit} to other untethered AR HMD 
%devices and develop methods to autonomously adapt performance-related parameters 
%with respect to the application and (or) user requirements.


%%%%%%%%%%%%%%%%%%%%%%%%%%%%%%%%% 80 CHAR %%%%%%%%%%%%%%%%%%%%%%%%%%%%%%%%%%%%%%


% \section*{Acknowledgements}
% %
% We thank our shepherd, Dr. Branislav Kusy, and the anonymous reviewers for
% their helpful feedback. We would also like to thank the clinical staff team 
% at the Ajou University Hospital Trauma Center, especially, Ms. Kyungjin Hwang, 
% Dr. Jayun Cho, Dr. Kyoungwon Jung, for their great help in designing,
% installing and maintaining {\myit}.
% %
% This work was partially supported by the Ministry of Trade, Industry and 
% Energy (MOTIE) and Korea Institute for Advancement of Technology (KIAT) 
% through the International Cooperative R\&D program, by the Korea Health
% Technology R\&D Project through KHIDI funded by the Ministry of Health 
% \& Welfare, Korea (HI16C0982), by the Institute for Information \& 
% communications Technology Promotion(IITP) grant funded by the Ministry
% of Science and ICT (2017-0-00526) and by the 
% % thanks - jpaek
% Basic Science Research Program through the National Research         
% Foundation of Korea (NRF) funded by the Ministry of Education        
% (NRF-2017R1D1A1B03031348).
